\documentclass{article}

\usepackage{graphicx}
\usepackage{amsmath}	
	
\title{Design \& Professional Skills\\
  Assignment 2: Debugging the {\em Bomber} Game}
\author{}
\date{}
	
\begin{document}

\maketitle

\section*{Instructions}

In the {\tt mhandley/ENGF2-2020} github repository there are three
Python programs in the {\tt Assignments/assignment2} directory.  These
are three implementations of exactly the same very simple very dumb
{\em Bomber} game. This game is a re-implementation of a game written for
the Sinclair ZX81 which originally ran in 1 KByte of RAM.

The objective of the game is to level the city by dropping bombs on
the buildings so you can land the plane.  The plane gets lower each
time it crosses the screen.  If you fly into a building, you die.  If
you succeed in flattening all the buildings, you can land the plane
and score 1000 points.  Your reward is to start all over again, but
with narrower buildings that are harder to hit.  

The three versions of the game should behave identically, but are written in different programming styles:
\begin{itemize}
\item {\tt bomber\_bigloop.py} uses a lot of global variables and one big main loop.
\item{\tt bomber\_proc.py} uses a procedural style, with fewer global variables and breaks the code into many functions.  It makes extensive use of python lists to pass data into and out of these functions.
\item{\tt bomber\_oo.py} uses an object-oriented style, defining classes for the plane, buildings, etc, and functions that operate on those classes.
\end{itemize}
The game requires python 3 and tkinter.

The problem is that the games have bugs.  \textbf{All three games have the same five bugs}.  Your task is:
\begin{enumerate}
\item Play the game.
\item Find a bug.
\item Write a brief bug report describing the bug.
\item Identify the cause of the bug.  Write a brief summary of the cause.
\item Fix the bug.
\item Repeat from 1.
\end{enumerate}

\subsection*{Bug Reports}

A bug report should be brief and to the point.  It should include:
\begin{itemize}
\item One sentence summary of the bug.
\item Description of what happens.
\item Description of what you think should happen.
\item Instructions for how to reproduce the bug.
\end{itemize}

\subsection*{Understanding the bug}

Once you've written the bug report, look at the code.  You can look at
all three versions, or just one version --- whatever you find easiest.

Identify what the code is doing when the bug is triggered.  Sometimes
the cause may be obvious from reading the code.  Often the cause is
not obvious, and even the flow of the code may not be obvious.  Then
you will need to instrument the code to figure out what it is doing.
In this case, I just want you to instrument the code using
\texttt{print()} - there's no need to use a debugger.  Generally, you
want to instrument the code without changing its behaviour until you
gather enough information to understand what the code is actually
doing (and hence why it differs from what it should be doing).

Another common debugging technique is to reduce the code complexity by
removing code to reduce it to the simplest case that still exhibits
the buggy behaviour.  This is a valuable technique when a bug is hard
to reproduce.

Another technique is to make things more deterministic.  For example,
in this game, the buildings are randomly generated heights.  You might
call \texttt{random.seed(x)} with a constant parameter \texttt{x}, so
the buildings are always generated the same way.  By manually choosing
different values of \texttt{x} you can settle on one that makes it
simpler to generate the buggy behaviour in a repeatable manner.  Of
course you'd then need to remove the fixed seed later, when you've fixed
the bug.

In general, you're hunting for evidence until you've found out what
the program is actually doing.  Only when you understand what the
program is doing should you think about how to fix it.

\subsection{Fixing the bugs}

These bugs are very simple.  Some are one line fixes, none requires
more than about three lines of extra code to fix.  \textbf{You only
  need to fix the bugs in one of the three versions of the code}, but
you can choose which version you work with.

\subsection*{Assessment}

This assignment is not assessed.  The purpose of the assignment is to
give you practice reading, understanding, and debugging a non-trivial
piece of code.  You many work with your friends on this assignment if
you wish.  You must hand in a reasonable attempt via Moodle to receive
a binary mark, as we want to see how you are progressing.  Hand in a
zipfile containing the text of your bug reports, you brief
explanations of the causes of the bugs, and the python source code of
\textbf{one} of the three versions of the game with the bugs fixed.

\subsection*{Optional Extra}

A few members of the class are more experienced programmers.  If you
find fixing the bugs easy, once you've fixed them, consider extending
the game.  For example, the bomb does not obey any reasonable physics
at the moment.  The game would be better if it fell in a more
realistic manner.  There will be a separate submission area on Moodle
for game extensions.  You won't get any extra marks, but we are
curious what you can achieve.
\end{document}
